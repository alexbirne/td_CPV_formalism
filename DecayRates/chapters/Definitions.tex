\section{Definitions}

In the following we define the two flavour eigenstates as \Bz and \Bzb, the two finalstates $f$ and $\overline{f}$ and the two mass eigenstates as
$\B_H$ (heavy) and $\B_L$ (light). Starting from this we define the four possible transitions
\begin{align}
   A_f &= \bra{f}T\ket{\Bz}\\
   \overline{A}_f &= \bra{f}T\ket{\Bzb} \\
   \overline{A}_{\overline{f}} &= \bra{\overline{f}}T\ket{\Bzb}\\
   A_{\overline{f}} &= \bra{\overline{f}}T\ket{\Bz}.
\end{align}
Last the quantities $\lambda$ and $\overline{\lambda}$ will be uses as follows:
\begin{equation}
    \lambda = \frac{q}{p}\frac{\overline{A}_f}{A_f} \hspace{0.5cm} \text{and} \hspace{0.5cm} \overline{\lambda}
    = \frac{p}{q}\frac{A_{\overline{f}}}{\overline{A}_{\overline{f}}}
\end{equation}
For the differences between the light and heavy mass eigenstates in the \B system we define the mass difference $\Delta m$ to be positive:
\begin{align}
    \Delta m = m_H - m_L \hspace{0.5cm} &\text{and} \hspace{0.5cm} m = \frac{1}{2}\left(m_H+ m_L\right)\label{eq:defmassdiff}\\
    \Delta \Gamma = \Gamma_H - \Gamma_L \hspace{0.5cm} &\text{and} \hspace{0.5cm} \Gamma = \frac{1}{2}\left(\Gamma_H + \Gamma_L\right)\label{eq:defdecratewidth}
\end{align}
