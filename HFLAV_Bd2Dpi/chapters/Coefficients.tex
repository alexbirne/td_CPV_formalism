\section{Coefficients}
\label{sec:coef}

In this second part we start with the definitions of the coefficients directly as they are given in the different mentioned publications.
We will use the following substitutions:
\begin{align*}
r &= \frac{\left|\overline{\kern -2.5pt A\kern -1.0pt}_{\kern 0.2pt D^{-}\pi^{+}}\right|}{\left|A_{\kern -0.8pt D^{-}\pi^{+}}\right|} = \frac{\left|A_{\kern -0.8pt D^{+}\pi^{-}}\right|}{\left|\overline{\kern -2.5pt A\kern -1.0pt}_{\kern 0.2pt D^{+}\pi^{-}}\right|}\\
R_f &= \frac{r}{1+r^2}
\end{align*}

\subsection{HFLAV}

Following Eqs. (129) and (130) and more explicitly for the coefficients $\Sf$ and $\Sfbar$ \mbox{Eq. (134)} in Ref.~\cite{Amhis:2016xyh} the \CP
coefficients are defined as follows:
\begin{align}
\Sf &= -2R_f\sin\!\left(\phi_{\text{mix}}+\phi_{\text{dec}}-\delta_{f}\right) = -2R_f\sin\!\left(2\beta+\gamma-\delta_{f}\right)\\
\Sfbar &= -2R_f\sin\!\left(\phi_{\text{mix}}+\phi_{\text{dec}}+\delta_{f}\right) = -2R_f\sin\!\left(2\beta+\gamma+\delta_{f}\right)\\
\Cf &= -\Cfbar = \frac{1-r^2}{1+r^2}
\end{align}

\subsection{\babar}

In Ref.~\cite{Aubert:2006tw} the \CP coefficients are defined as follows (Eqs. (2) and (3)):
\begin{align}
S_{\pm}=-\frac{2\text{Im}\left(\lambda_{\pm}\right)}{1+\left|\lambda_{\pm}\right|^2}\\
C=\frac{1-r^2}{1+r^2}
\end{align}
where
\begin{equation}
\lambda_\pm=r e^{-i\left(2\beta+\gamma\mp\delta_f\right)}
\end{equation}
With $\text{Im}\left(\lambda_\pm\right) = r\sin\!\left(-2\beta-\gamma\pm\delta_f\right)$ we get the following for the \CP coefficients:
\begin{align}
S_{+}&=-\frac{2r\sin\!\left(-2\beta-\gamma+\delta_f\right)}{1+r^2}=2R_f\sin\!\left(2\beta+\gamma-\delta_f\right)\\
S_{-}&=-\frac{2r\sin\!\left(-2\beta-\gamma-\delta_f\right)}{1+r^2}=2R_f\sin\!\left(2\beta+\gamma+\delta_f\right)\\
C&=\frac{1-r^2}{1+r^2}
\end{align}

\subsection{\belle}

In the publication of \belle \cite{Ronga:2006hv} the \CP coefficients are defined in Eq. (2) as
\begin{align}
S^{\pm}&=\frac{2\left(-1\right)^{L}r\sin\!\left(2\beta+\gamma\pm\delta_f\right)}{1+r^2}\\
C&=\frac{1-r^2}{1+r^2}
\end{align}
where $L$ is the orbital angular momentum of the final state and therefore for the finalstate $D\pi$ $L=0$ applies (note that for
the CKM angles the substitution $\phi_1=\beta$ and $\phi_3=\gamma$  was implicitly done here). For the explicit \CP coefficients
we get now get the following:
\begin{align}
S^{-}&=\frac{2r\sin\!\left(2\beta+\gamma-\delta_f\right)}{1+r^2} = 2R_f\sin\!\left(2\beta+\gamma-\delta_f\right)\\
S^{+}&=\frac{2r\sin\!\left(2\beta+\gamma+\delta_f\right)}{1+r^2} = 2R_f\sin\!\left(2\beta+\gamma+\delta_f\right)\\
C&=\frac{1-r^2}{1+r^2}
\end{align}

\subsection{Summary}

Comparing the given \CP coefficients above we derive the following transformation between the HFLAV convention and the
notations used by the \babar and \belle collaborations:
\begin{align}
\Sf &= -S_{+}(\text{\babar}) = -S^{-}(\text{\belle})\\
\Sfbar &= -S_{-}(\text{\babar}) = -S^{+}(\text{\belle})\\
\Cf &= -\Cfbar = C(\text{\babar}) = C(\text{\belle})
\end{align}
